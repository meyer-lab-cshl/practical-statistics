\PassOptionsToPackage{unicode=true}{hyperref} % options for packages loaded elsewhere
\PassOptionsToPackage{hyphens}{url}
%
\documentclass[]{article}
\usepackage{lmodern}
\usepackage{amssymb,amsmath}
\usepackage{ifxetex,ifluatex}
\usepackage{fixltx2e} % provides \textsubscript
\ifnum 0\ifxetex 1\fi\ifluatex 1\fi=0 % if pdftex
  \usepackage[T1]{fontenc}
  \usepackage[utf8]{inputenc}
  \usepackage{textcomp} % provides euro and other symbols
\else % if luatex or xelatex
  \usepackage{unicode-math}
  \defaultfontfeatures{Ligatures=TeX,Scale=MatchLowercase}
\fi
% use upquote if available, for straight quotes in verbatim environments
\IfFileExists{upquote.sty}{\usepackage{upquote}}{}
% use microtype if available
\IfFileExists{microtype.sty}{%
\usepackage[]{microtype}
\UseMicrotypeSet[protrusion]{basicmath} % disable protrusion for tt fonts
}{}
\IfFileExists{parskip.sty}{%
\usepackage{parskip}
}{% else
\setlength{\parindent}{0pt}
\setlength{\parskip}{6pt plus 2pt minus 1pt}
}
\usepackage{hyperref}
\hypersetup{
            pdftitle={Practical Statistics for Experimental Biologists in RStudio},
            pdfauthor={Caroline Walter},
            pdfborder={0 0 0},
            breaklinks=true}
\urlstyle{same}  % don't use monospace font for urls
\usepackage[margin=1in]{geometry}
\usepackage{color}
\usepackage{fancyvrb}
\newcommand{\VerbBar}{|}
\newcommand{\VERB}{\Verb[commandchars=\\\{\}]}
\DefineVerbatimEnvironment{Highlighting}{Verbatim}{commandchars=\\\{\}}
% Add ',fontsize=\small' for more characters per line
\usepackage{framed}
\definecolor{shadecolor}{RGB}{248,248,248}
\newenvironment{Shaded}{\begin{snugshade}}{\end{snugshade}}
\newcommand{\AlertTok}[1]{\textcolor[rgb]{0.94,0.16,0.16}{#1}}
\newcommand{\AnnotationTok}[1]{\textcolor[rgb]{0.56,0.35,0.01}{\textbf{\textit{#1}}}}
\newcommand{\AttributeTok}[1]{\textcolor[rgb]{0.77,0.63,0.00}{#1}}
\newcommand{\BaseNTok}[1]{\textcolor[rgb]{0.00,0.00,0.81}{#1}}
\newcommand{\BuiltInTok}[1]{#1}
\newcommand{\CharTok}[1]{\textcolor[rgb]{0.31,0.60,0.02}{#1}}
\newcommand{\CommentTok}[1]{\textcolor[rgb]{0.56,0.35,0.01}{\textit{#1}}}
\newcommand{\CommentVarTok}[1]{\textcolor[rgb]{0.56,0.35,0.01}{\textbf{\textit{#1}}}}
\newcommand{\ConstantTok}[1]{\textcolor[rgb]{0.00,0.00,0.00}{#1}}
\newcommand{\ControlFlowTok}[1]{\textcolor[rgb]{0.13,0.29,0.53}{\textbf{#1}}}
\newcommand{\DataTypeTok}[1]{\textcolor[rgb]{0.13,0.29,0.53}{#1}}
\newcommand{\DecValTok}[1]{\textcolor[rgb]{0.00,0.00,0.81}{#1}}
\newcommand{\DocumentationTok}[1]{\textcolor[rgb]{0.56,0.35,0.01}{\textbf{\textit{#1}}}}
\newcommand{\ErrorTok}[1]{\textcolor[rgb]{0.64,0.00,0.00}{\textbf{#1}}}
\newcommand{\ExtensionTok}[1]{#1}
\newcommand{\FloatTok}[1]{\textcolor[rgb]{0.00,0.00,0.81}{#1}}
\newcommand{\FunctionTok}[1]{\textcolor[rgb]{0.00,0.00,0.00}{#1}}
\newcommand{\ImportTok}[1]{#1}
\newcommand{\InformationTok}[1]{\textcolor[rgb]{0.56,0.35,0.01}{\textbf{\textit{#1}}}}
\newcommand{\KeywordTok}[1]{\textcolor[rgb]{0.13,0.29,0.53}{\textbf{#1}}}
\newcommand{\NormalTok}[1]{#1}
\newcommand{\OperatorTok}[1]{\textcolor[rgb]{0.81,0.36,0.00}{\textbf{#1}}}
\newcommand{\OtherTok}[1]{\textcolor[rgb]{0.56,0.35,0.01}{#1}}
\newcommand{\PreprocessorTok}[1]{\textcolor[rgb]{0.56,0.35,0.01}{\textit{#1}}}
\newcommand{\RegionMarkerTok}[1]{#1}
\newcommand{\SpecialCharTok}[1]{\textcolor[rgb]{0.00,0.00,0.00}{#1}}
\newcommand{\SpecialStringTok}[1]{\textcolor[rgb]{0.31,0.60,0.02}{#1}}
\newcommand{\StringTok}[1]{\textcolor[rgb]{0.31,0.60,0.02}{#1}}
\newcommand{\VariableTok}[1]{\textcolor[rgb]{0.00,0.00,0.00}{#1}}
\newcommand{\VerbatimStringTok}[1]{\textcolor[rgb]{0.31,0.60,0.02}{#1}}
\newcommand{\WarningTok}[1]{\textcolor[rgb]{0.56,0.35,0.01}{\textbf{\textit{#1}}}}
\usepackage{graphicx,grffile}
\makeatletter
\def\maxwidth{\ifdim\Gin@nat@width>\linewidth\linewidth\else\Gin@nat@width\fi}
\def\maxheight{\ifdim\Gin@nat@height>\textheight\textheight\else\Gin@nat@height\fi}
\makeatother
% Scale images if necessary, so that they will not overflow the page
% margins by default, and it is still possible to overwrite the defaults
% using explicit options in \includegraphics[width, height, ...]{}
\setkeys{Gin}{width=\maxwidth,height=\maxheight,keepaspectratio}
\setlength{\emergencystretch}{3em}  % prevent overfull lines
\providecommand{\tightlist}{%
  \setlength{\itemsep}{0pt}\setlength{\parskip}{0pt}}
\setcounter{secnumdepth}{5}
% Redefines (sub)paragraphs to behave more like sections
\ifx\paragraph\undefined\else
\let\oldparagraph\paragraph
\renewcommand{\paragraph}[1]{\oldparagraph{#1}\mbox{}}
\fi
\ifx\subparagraph\undefined\else
\let\oldsubparagraph\subparagraph
\renewcommand{\subparagraph}[1]{\oldsubparagraph{#1}\mbox{}}
\fi

% set default figure placement to htbp
\makeatletter
\def\fps@figure{htbp}
\makeatother


\title{Practical Statistics for Experimental Biologists in RStudio}
\author{Caroline Walter}
\date{August 2020}

\begin{document}
\maketitle

{
\setcounter{tocdepth}{3}
\tableofcontents
}
\hypertarget{background}{%
\section{Background}\label{background}}

This course will be reviewing some practical statistics work in the R
computing language useful in experimental biology.

Below are the necessary libraries for the statistical work we will be
doing.

\begin{Shaded}
\begin{Highlighting}[]
\CommentTok{#Loading libraries}
\KeywordTok{library}\NormalTok{(caret)}
\KeywordTok{library}\NormalTok{(tidyverse)}
\KeywordTok{library}\NormalTok{(dplyr)}
\KeywordTok{library}\NormalTok{(RDocumentation)}
\KeywordTok{library}\NormalTok{(psych)}
\KeywordTok{library}\NormalTok{(car)}
\KeywordTok{library}\NormalTok{(pwr)}
\end{Highlighting}
\end{Shaded}

\hypertarget{contingecy-tables}{%
\section{Contingecy Tables}\label{contingecy-tables}}

\begin{itemize}
\tightlist
\item
  create a fake data set
\end{itemize}

\begin{Shaded}
\begin{Highlighting}[]
\NormalTok{fk_data <-}\StringTok{ }\KeywordTok{data.frame}\NormalTok{(}\DataTypeTok{x1 =} \KeywordTok{sample}\NormalTok{(letters[}\DecValTok{1}\OperatorTok{:}\DecValTok{5}\NormalTok{], }\DecValTok{20}\NormalTok{, }\DataTypeTok{replace=}\OtherTok{TRUE}\NormalTok{), }
                      \DataTypeTok{x2 =} \KeywordTok{sample}\NormalTok{(LETTERS[}\DecValTok{1}\OperatorTok{:}\DecValTok{5}\NormalTok{], }\DecValTok{20}\NormalTok{, }\DataTypeTok{replace =} \OtherTok{TRUE}\NormalTok{))}
\end{Highlighting}
\end{Shaded}

\begin{itemize}
\tightlist
\item
  have a look at the data set
\end{itemize}

\begin{Shaded}
\begin{Highlighting}[]
\KeywordTok{print.data.frame}\NormalTok{(fk_data)}
\end{Highlighting}
\end{Shaded}

\begin{verbatim}
##    x1 x2
## 1   b  D
## 2   a  E
## 3   c  B
## 4   a  A
## 5   b  C
## 6   a  A
## 7   d  B
## 8   e  C
## 9   b  D
## 10  c  E
## 11  a  E
## 12  b  A
## 13  b  A
## 14  b  A
## 15  b  B
## 16  c  A
## 17  d  A
## 18  b  D
## 19  a  C
## 20  e  D
\end{verbatim}

\begin{itemize}
\tightlist
\item
  create a table
\end{itemize}

\begin{Shaded}
\begin{Highlighting}[]
\NormalTok{my_table_}\DecValTok{0}\NormalTok{ <-}\StringTok{ }\KeywordTok{table}\NormalTok{(fk_data}\OperatorTok{$}\NormalTok{x1, fk_data}\OperatorTok{$}\NormalTok{x2)}
\KeywordTok{print.table}\NormalTok{(my_table_}\DecValTok{0}\NormalTok{)}
\end{Highlighting}
\end{Shaded}

\begin{verbatim}
##    
##     A B C D E
##   a 2 0 1 0 2
##   b 3 1 1 3 0
##   c 1 1 0 0 1
##   d 1 1 0 0 0
##   e 0 0 1 1 0
\end{verbatim}

\begin{itemize}
\tightlist
\item
  if we want to have row and column totals
\end{itemize}

\begin{Shaded}
\begin{Highlighting}[]
\NormalTok{my_table_}\DecValTok{01}\NormalTok{<-}\StringTok{ }\KeywordTok{addmargins}\NormalTok{(my_table_}\DecValTok{0}\NormalTok{)}
\KeywordTok{print.table}\NormalTok{(my_table_}\DecValTok{01}\NormalTok{)}
\end{Highlighting}
\end{Shaded}

\begin{verbatim}
##      
##        A  B  C  D  E Sum
##   a    2  0  1  0  2   5
##   b    3  1  1  3  0   8
##   c    1  1  0  0  1   3
##   d    1  1  0  0  0   2
##   e    0  0  1  1  0   2
##   Sum  7  3  3  4  3  20
\end{verbatim}

\begin{itemize}
\tightlist
\item
  convert it to a dataframe
\end{itemize}

\begin{Shaded}
\begin{Highlighting}[]
\NormalTok{my_table_}\DecValTok{1}\NormalTok{ <-}\StringTok{ }\KeywordTok{as.data.frame.matrix}\NormalTok{(my_table_}\DecValTok{0}\NormalTok{)}
\end{Highlighting}
\end{Shaded}

\begin{itemize}
\tightlist
\item
  have a look at the table
\end{itemize}

\begin{Shaded}
\begin{Highlighting}[]
\KeywordTok{print.data.frame}\NormalTok{(my_table_}\DecValTok{1}\NormalTok{)}
\end{Highlighting}
\end{Shaded}

\begin{verbatim}
##   A B C D E
## a 2 0 1 0 2
## b 3 1 1 3 0
## c 1 1 0 0 1
## d 1 1 0 0 0
## e 0 0 1 1 0
\end{verbatim}

\begin{itemize}
\tightlist
\item
  to have a table of proportions based on rows, and convert it to a
  dataframe
\end{itemize}

\begin{Shaded}
\begin{Highlighting}[]
\NormalTok{my_table_}\DecValTok{2}\NormalTok{ <-}\StringTok{ }\KeywordTok{prop.table}\NormalTok{(my_table_}\DecValTok{0}\NormalTok{, }\DataTypeTok{margin =} \DecValTok{1}\NormalTok{) }\OperatorTok
\StringTok{  }\KeywordTok{as.data.frame.matrix}\NormalTok{() }
\end{Highlighting}
\end{Shaded}

\begin{itemize}
\tightlist
\item
  have a look at the table
\end{itemize}

\begin{Shaded}
\begin{Highlighting}[]
\KeywordTok{print.data.frame}\NormalTok{(my_table_}\DecValTok{2}\NormalTok{, }\DataTypeTok{digits =} \DecValTok{2}\NormalTok{)}
\end{Highlighting}
\end{Shaded}

\begin{verbatim}
##      A    B    C    D    E
## a 0.40 0.00 0.20 0.00 0.40
## b 0.38 0.12 0.12 0.38 0.00
## c 0.33 0.33 0.00 0.00 0.33
## d 0.50 0.50 0.00 0.00 0.00
## e 0.00 0.00 0.50 0.50 0.00
\end{verbatim}

\begin{itemize}
\tightlist
\item
  to have a table of proportions based on columns
\end{itemize}

\begin{Shaded}
\begin{Highlighting}[]
\NormalTok{my_table_}\DecValTok{3}\NormalTok{ <-}\StringTok{ }\KeywordTok{prop.table}\NormalTok{(my_table_}\DecValTok{0}\NormalTok{, }\DataTypeTok{margin=}\DecValTok{2}\NormalTok{)}\OperatorTok
\StringTok{  }\KeywordTok{as.data.frame.matrix}\NormalTok{() }
\end{Highlighting}
\end{Shaded}

\begin{itemize}
\tightlist
\item
  have a look at the table
\end{itemize}

\begin{Shaded}
\begin{Highlighting}[]
\KeywordTok{print.data.frame}\NormalTok{(my_table_}\DecValTok{2}\NormalTok{, }\DataTypeTok{digits =} \DecValTok{2}\NormalTok{)}
\end{Highlighting}
\end{Shaded}

\begin{verbatim}
##      A    B    C    D    E
## a 0.40 0.00 0.20 0.00 0.40
## b 0.38 0.12 0.12 0.38 0.00
## c 0.33 0.33 0.00 0.00 0.33
## d 0.50 0.50 0.00 0.00 0.00
## e 0.00 0.00 0.50 0.50 0.00
\end{verbatim}

\hypertarget{sensitivity-and-specificity}{%
\section{Sensitivity and
Specificity}\label{sensitivity-and-specificity}}

For the AIP test:

\begin{itemize}
\tightlist
\item
  Sensitivity: (AIP example) the probability of testing positive given
  that the subjevt has the disease
\end{itemize}

\(Sensitivity = p(test^+|disease^+)=0.82\)

\begin{itemize}
\tightlist
\item
  Specificity: (AIP example) the probability of a negative test given
  that the subject does not have the disease
\end{itemize}

\(Specificity = p(test^-|disease^-)=0.963\)

\begin{itemize}
\tightlist
\item
  here is a helpful youtube video:
  \url{https://www.youtube.com/watch?v=9f5XgjWpzi0}
\end{itemize}

\begin{Shaded}
\begin{Highlighting}[]
\NormalTok{devtools}\OperatorTok{::}\KeywordTok{install_github}\NormalTok{(}\StringTok{"datacamp/RDocumentation"}\NormalTok{)}

\NormalTok{data01 <-}\StringTok{ }\KeywordTok{factor}\NormalTok{(}\KeywordTok{c}\NormalTok{(}\StringTok{"A"}\NormalTok{, }\StringTok{"B"}\NormalTok{, }\StringTok{"B"}\NormalTok{, }\StringTok{"B"}\NormalTok{))}
\NormalTok{data02 <-}\StringTok{ }\KeywordTok{factor}\NormalTok{(}\KeywordTok{c}\NormalTok{(}\StringTok{"A"}\NormalTok{, }\StringTok{"B"}\NormalTok{, }\StringTok{"B"}\NormalTok{, }\StringTok{"B"}\NormalTok{))}

\NormalTok{ref01 <-}\StringTok{ }\KeywordTok{factor}\NormalTok{(}\KeywordTok{c}\NormalTok{(}\StringTok{"B"}\NormalTok{, }\StringTok{"B"}\NormalTok{, }\StringTok{"B"}\NormalTok{, }\StringTok{"B"}\NormalTok{))}
\NormalTok{ref02 <-}\StringTok{ }\KeywordTok{factor}\NormalTok{(}\KeywordTok{c}\NormalTok{(}\StringTok{"B"}\NormalTok{, }\StringTok{"A"}\NormalTok{, }\StringTok{"B"}\NormalTok{, }\StringTok{"B"}\NormalTok{))}
\end{Highlighting}
\end{Shaded}

\hypertarget{positive-predictive-value-ppv}{%
\section{Positive Predictive Value
(PPV)}\label{positive-predictive-value-ppv}}

Positive Predictive Value is the probability that subjects with a
positive screening test truly have the disease

\(p(disease^+|test^+) = \frac{TP}{TP+FP}=\frac{82}{82+36,996}=0.22\)

Even if you test positive, the probablity of you having AIP is still
very low.

\begin{Shaded}
\begin{Highlighting}[]
\KeywordTok{table}\NormalTok{(data01, ref01)}
\end{Highlighting}
\end{Shaded}

\begin{verbatim}
##       ref01
## data01 B
##      A 1
##      B 3
\end{verbatim}

\begin{Shaded}
\begin{Highlighting}[]
\KeywordTok{sensitivity}\NormalTok{(data01, ref01) }
\end{Highlighting}
\end{Shaded}

\begin{verbatim}
## [1] 0.75
\end{verbatim}

\begin{Shaded}
\begin{Highlighting}[]
\KeywordTok{posPredValue}\NormalTok{(data01, ref01) }
\end{Highlighting}
\end{Shaded}

\begin{verbatim}
## [1] NA
\end{verbatim}

\begin{Shaded}
\begin{Highlighting}[]
\KeywordTok{table}\NormalTok{(data02, ref02)}
\end{Highlighting}
\end{Shaded}

\begin{verbatim}
##       ref02
## data02 A B
##      A 0 1
##      B 1 2
\end{verbatim}

\begin{Shaded}
\begin{Highlighting}[]
\KeywordTok{sensitivity}\NormalTok{(data02, ref02) }
\end{Highlighting}
\end{Shaded}

\begin{verbatim}
## [1] 0
\end{verbatim}

\begin{Shaded}
\begin{Highlighting}[]
\KeywordTok{posPredValue}\NormalTok{(data02, ref02) }
\end{Highlighting}
\end{Shaded}

\begin{verbatim}
## [1] 0
\end{verbatim}

\begin{Shaded}
\begin{Highlighting}[]
\NormalTok{data03 <-}\StringTok{ }\KeywordTok{factor}\NormalTok{(}\KeywordTok{c}\NormalTok{(}\StringTok{"A"}\NormalTok{, }\StringTok{"B"}\NormalTok{, }\StringTok{"B"}\NormalTok{, }\StringTok{"B"}\NormalTok{))}
\NormalTok{data04 <-}\StringTok{ }\KeywordTok{factor}\NormalTok{(}\KeywordTok{c}\NormalTok{(}\StringTok{"B"}\NormalTok{, }\StringTok{"B"}\NormalTok{, }\StringTok{"B"}\NormalTok{, }\StringTok{"B"}\NormalTok{))}

\NormalTok{ref03 <-}\StringTok{ }\KeywordTok{factor}\NormalTok{(}\KeywordTok{c}\NormalTok{(}\StringTok{"B"}\NormalTok{, }\StringTok{"B"}\NormalTok{, }\StringTok{"B"}\NormalTok{, }\StringTok{"B"}\NormalTok{), }\DataTypeTok{levels =} \KeywordTok{c}\NormalTok{(}\StringTok{"A"}\NormalTok{, }\StringTok{"B"}\NormalTok{))}
\NormalTok{ref04 <-}\StringTok{ }\KeywordTok{factor}\NormalTok{(}\KeywordTok{c}\NormalTok{(}\StringTok{"B"}\NormalTok{, }\StringTok{"A"}\NormalTok{, }\StringTok{"B"}\NormalTok{, }\StringTok{"B"}\NormalTok{))}
\end{Highlighting}
\end{Shaded}

\hypertarget{negative-predictive-value-npv}{%
\section{Negative Predictive Value
(NPV)}\label{negative-predictive-value-npv}}

Begative predictive value is the probability that subjects with a
negative screening test truly do not have the disease.

\(p(disease^-|test^-)\)

\begin{Shaded}
\begin{Highlighting}[]
\KeywordTok{table}\NormalTok{(data03, ref03)}
\end{Highlighting}
\end{Shaded}

\begin{verbatim}
##       ref03
## data03 A B
##      A 0 1
##      B 0 3
\end{verbatim}

\begin{Shaded}
\begin{Highlighting}[]
\KeywordTok{specificity}\NormalTok{(data03, ref03) }\CommentTok{#0.75}
\end{Highlighting}
\end{Shaded}

\begin{verbatim}
## [1] 0.75
\end{verbatim}

\begin{Shaded}
\begin{Highlighting}[]
\KeywordTok{negPredValue}\NormalTok{(data03, ref03) }
\end{Highlighting}
\end{Shaded}

\begin{verbatim}
## [1] NA
\end{verbatim}

\begin{Shaded}
\begin{Highlighting}[]
\KeywordTok{table}\NormalTok{(data04, ref04)}
\end{Highlighting}
\end{Shaded}

\begin{verbatim}
##       ref04
## data04 A B
##      B 1 3
\end{verbatim}

\begin{Shaded}
\begin{Highlighting}[]
\KeywordTok{specificity}\NormalTok{(data04, ref04) }\CommentTok{#1}
\end{Highlighting}
\end{Shaded}

\begin{verbatim}
## [1] 1
\end{verbatim}

\begin{Shaded}
\begin{Highlighting}[]
\KeywordTok{negPredValue}\NormalTok{(data04, ref04) }\CommentTok{#NaN}
\end{Highlighting}
\end{Shaded}

\begin{verbatim}
## [1] NaN
\end{verbatim}

\begin{Shaded}
\begin{Highlighting}[]
\ControlFlowTok{if}\NormalTok{(}\OperatorTok{!}\KeywordTok{isTRUE}\NormalTok{(}\KeywordTok{all.equal}\NormalTok{(}\KeywordTok{sensitivity}\NormalTok{(data01, ref01), }\FloatTok{.75}\NormalTok{))) }\KeywordTok{stop}\NormalTok{(}\StringTok{"error in sensitivity test 1"}\NormalTok{)}
\ControlFlowTok{if}\NormalTok{(}\OperatorTok{!}\KeywordTok{isTRUE}\NormalTok{(}\KeywordTok{all.equal}\NormalTok{(}\KeywordTok{sensitivity}\NormalTok{(data02, ref02), }\DecValTok{0}\NormalTok{))) }\KeywordTok{stop}\NormalTok{(}\StringTok{"error in sensitivity test 2"}\NormalTok{)}
\NormalTok{ref03 <-}\StringTok{ }\KeywordTok{factor}\NormalTok{(}\KeywordTok{c}\NormalTok{(}\StringTok{"B"}\NormalTok{, }\StringTok{"B"}\NormalTok{, }\StringTok{"B"}\NormalTok{, }\StringTok{"B"}\NormalTok{))}
\ControlFlowTok{if}\NormalTok{(}\OperatorTok{!}\KeywordTok{is.na}\NormalTok{(}\KeywordTok{sensitivity}\NormalTok{(data02, ref03, }\StringTok{"A"}\NormalTok{))) }\KeywordTok{stop}\NormalTok{(}\StringTok{"error in sensitivity test3"}\NormalTok{)}
      
   \KeywordTok{options}\NormalTok{(}\DataTypeTok{show.error.messages =} \OtherTok{FALSE}\NormalTok{)}
\NormalTok{   test1 <-}\KeywordTok{try}\NormalTok{(}\KeywordTok{sensitivity}\NormalTok{(data02, }\KeywordTok{as.character}\NormalTok{(ref03)))}
   \ControlFlowTok{if}\NormalTok{(}\KeywordTok{grep}\NormalTok{(}\StringTok{"Error"}\NormalTok{, test1) }\OperatorTok{!=}\StringTok{ }\DecValTok{1}\NormalTok{)}
      \KeywordTok{stop}\NormalTok{(}\StringTok{"error in sensitivity calculation - allowed non-factors"}\NormalTok{)}
   \KeywordTok{options}\NormalTok{(}\DataTypeTok{show.error.messages =} \OtherTok{TRUE}\NormalTok{)}
   
\NormalTok{   ref03 <-}\StringTok{ }\KeywordTok{factor}\NormalTok{(}\KeywordTok{c}\NormalTok{(}\StringTok{"B"}\NormalTok{, }\StringTok{"B"}\NormalTok{, }\StringTok{"B"}\NormalTok{, }\StringTok{"B"}\NormalTok{), }\DataTypeTok{levels =} \KeywordTok{c}\NormalTok{(}\StringTok{"A"}\NormalTok{, }\StringTok{"B"}\NormalTok{))}
   
   \ControlFlowTok{if}\NormalTok{(}\OperatorTok{!}\KeywordTok{isTRUE}\NormalTok{(}\KeywordTok{all.equal}\NormalTok{(}\KeywordTok{specificity}\NormalTok{(data03, ref03), }\FloatTok{.75}\NormalTok{))) }\KeywordTok{stop}\NormalTok{(}\StringTok{"error in specificity test 1"}\NormalTok{)}

   \ControlFlowTok{if}\NormalTok{(}\OperatorTok{!}\KeywordTok{isTRUE}\NormalTok{(}\KeywordTok{all.equal}\NormalTok{(}\KeywordTok{specificity}\NormalTok{(data04, ref04), }\FloatTok{1.00}\NormalTok{))) }\KeywordTok{stop}\NormalTok{(}\StringTok{"error in specificity test 2"}\NormalTok{)}

   \ControlFlowTok{if}\NormalTok{(}\OperatorTok{!}\KeywordTok{is.na}\NormalTok{(}\KeywordTok{specificity}\NormalTok{(data01, ref01, }\StringTok{"A"}\NormalTok{))) }\KeywordTok{stop}\NormalTok{(}\StringTok{"error in specificity test3"}\NormalTok{)}
      
   \KeywordTok{options}\NormalTok{(}\DataTypeTok{show.error.messages =} \OtherTok{FALSE}\NormalTok{)}
\NormalTok{   test1 <-}\KeywordTok{try}\NormalTok{(}\KeywordTok{specificity}\NormalTok{(data04, }\KeywordTok{as.character}\NormalTok{(ref03)))}
   \ControlFlowTok{if}\NormalTok{(}\KeywordTok{grep}\NormalTok{(}\StringTok{"Error"}\NormalTok{, test1) }\OperatorTok{!=}\StringTok{ }\DecValTok{1}\NormalTok{)}
      \KeywordTok{stop}\NormalTok{(}\StringTok{"error in specificity calculation - allowed non-factors"}\NormalTok{)}
   \KeywordTok{options}\NormalTok{(}\DataTypeTok{show.error.messages =} \OtherTok{TRUE}\NormalTok{)}
\end{Highlighting}
\end{Shaded}

\hypertarget{prevalence}{%
\section{Prevalence}\label{prevalence}}

Prevalence is the fraction of individuals in a population who have a
disease. \(prevalence = p(disease^+) = 0.50\)

\begin{itemize}
\tightlist
\item
  Usage: prevalence(model, type = c(``pop'', ``bnp'', ``wnp''), i=NULL,
  \ldots{})
\end{itemize}

\begin{Shaded}
\begin{Highlighting}[]
\KeywordTok{set.seed}\NormalTok{(}\DecValTok{0934}\NormalTok{)}
\NormalTok{Data.All.df}\FloatTok{.2008}\NormalTok{ <-}\StringTok{ }\KeywordTok{data.frame}\NormalTok{(}\DataTypeTok{FSA =} \KeywordTok{sample}\NormalTok{(}\KeywordTok{c}\NormalTok{(}\StringTok{"N8N"}\NormalTok{, }\StringTok{"N8R"}\NormalTok{, }\StringTok{"B3L"}\NormalTok{, }\StringTok{"P1H"}\NormalTok{), }\DecValTok{50}\NormalTok{, T),}
                               \DataTypeTok{Lyme =} \KeywordTok{sample}\NormalTok{(}\DecValTok{0}\OperatorTok{:}\DecValTok{1}\NormalTok{, }\DecValTok{50}\NormalTok{, T),}
                               \DataTypeTok{stringsAsFactors =}\NormalTok{ F)}
\end{Highlighting}
\end{Shaded}

\begin{itemize}
\tightlist
\item
  First 10 observations:
\end{itemize}

\begin{Shaded}
\begin{Highlighting}[]
\KeywordTok{head}\NormalTok{(Data.All.df}\FloatTok{.2008}\NormalTok{)}
\end{Highlighting}
\end{Shaded}

\begin{verbatim}
##   FSA Lyme
## 1 N8N    1
## 2 P1H    1
## 3 N8N    0
## 4 P1H    0
## 5 N8N    1
## 6 N8N    1
\end{verbatim}

\begin{itemize}
\tightlist
\item
  Prevalence can be calculated as the number of positive diagnoses
  divided by the total number of observations, i.e.~sum(Lyme)/n().
\end{itemize}

\begin{Shaded}
\begin{Highlighting}[]
\NormalTok{Data.All.df}\FloatTok{.2008} \OperatorTok
\StringTok{  }\KeywordTok{group_by}\NormalTok{(FSA) }\OperatorTok
\StringTok{  }\KeywordTok{summarise}\NormalTok{(}\DataTypeTok{Prevalence =} \KeywordTok{sum}\NormalTok{(Lyme)}\OperatorTok{/}\KeywordTok{n}\NormalTok{())}
\end{Highlighting}
\end{Shaded}

\begin{verbatim}
## # A tibble: 4 x 2
##   FSA   Prevalence
##   <chr>      <dbl>
## 1 B3L        0.778
## 2 N8N        0.571
## 3 N8R        0.583
## 4 P1H        0.467
\end{verbatim}

\hypertarget{frequentist-vs.bayesian-statistics}{%
\section{Frequentist vs.~Bayesian
Statistics}\label{frequentist-vs.bayesian-statistics}}

\begin{itemize}
\tightlist
\item
  Frequentist Statistics (aka classical statistics) focusses on
  likelihood. It avoids calculations involving prior odds, and therefore
  yields results that are prone to misinterpretation due to the base
  rate fallacy. Frequentist statistics is used heavily in biological
  research. It can still be useful and informative if you know exactly
  what to watch out for.
\end{itemize}

\(p(data|hypothesis)\)

\begin{itemize}
\item
  Base Rate Fallacy: If presented with related base rate information
  (generic, general information) and specific information (information
  pertaining only to a certain case), the mind tends to ignore the forer
  and focus of the latter.
\item
  Iron Law of Frequentist Statistics: Never compute the probability of a
  hypothesis.
\item
  Bayesian Statistics explicitly accounts for prior odds. It focusses on
  computing posterior probabilities and requires prior information that
  is often hard to quantify. It is central to the modern machine
  learning and more advanced areas of quantitative biology. Experimental
  researchers in biology tend not to use Bayesian statistics.
\end{itemize}

\(p(hypothesis|data)\)

\hypertarget{fishers-exact-test}{%
\section{Fisher's Exact Test}\label{fishers-exact-test}}

Fisher's exact test is a statistical significane test used in the
analysis of contingency tables and in place of chi square test in 2 by 2
tables, especially in cases of small samples. The test is useful for
categorical data that result from classifying objects in two different
ways; it is used to examine the significance of the association
(contingency) between the two kinds of classification.

Example: A British woman claimed to be able to distinguish whether milk
or tea was added to the cup first. To test, she was given 8 cups of tea,
in four of which milk was added first. The null hypothesis is that there
is no association between the true order of pouring and the woman's
guess, the alternative that there is a positive association (that the
odds ratio is greater than 1).

\begin{itemize}
\tightlist
\item
  Syntax: fisher.test(x, y = NULL, workspace = 200000, hybrid = FALSE,
  \#hybridPars = c(expect = 5, percent = 80, Emin = 1), \#control =
  list(), or = 1, alternative = ``two.sided'', \#conf.int = TRUE,
  conf.level = 0.95, \#simulate.p.value = FALSE, B = 2000)
\end{itemize}

\begin{Shaded}
\begin{Highlighting}[]
\NormalTok{TeaTasting <-}
\StringTok{  }\KeywordTok{matrix}\NormalTok{(}\KeywordTok{c}\NormalTok{(}\DecValTok{3}\NormalTok{,}\DecValTok{1}\NormalTok{,}\DecValTok{1}\NormalTok{,}\DecValTok{3}\NormalTok{), }
         \DataTypeTok{nrow=}\DecValTok{2}\NormalTok{, }
         \DataTypeTok{dimnames =} \KeywordTok{list}\NormalTok{(}\DataTypeTok{Guess =} \KeywordTok{c}\NormalTok{(}\StringTok{"Milk"}\NormalTok{, }\StringTok{"Tea"}\NormalTok{), }
                         \DataTypeTok{Truth =} \KeywordTok{c}\NormalTok{(}\StringTok{"Milk"}\NormalTok{, }\StringTok{"Tea"}\NormalTok{)))}
\KeywordTok{fisher.test}\NormalTok{(TeaTasting, }\DataTypeTok{alternative =} \StringTok{"greater"}\NormalTok{)}
\end{Highlighting}
\end{Shaded}

\begin{verbatim}
## 
##  Fisher's Exact Test for Count Data
## 
## data:  TeaTasting
## p-value = 0.2429
## alternative hypothesis: true odds ratio is greater than 1
## 95 percent confidence interval:
##  0.3135693       Inf
## sample estimates:
## odds ratio 
##   6.408309
\end{verbatim}

\hypertarget{bernoulli-distribution}{%
\section{Bernoulli Distribution}\label{bernoulli-distribution}}

Bernoulli Distribution: a discrete distribution having two possible
outcomes labeled by n=0 and n=1 in which n=1(``success'') occurs with
probability p and n=0 (``failure'') occurs with porbability q = 1-p,
where 0\textless{}p\textless{}1. Describes probabilities for a binary
variable. (Biased coin example); when the probabilities sum up to 100\%.

Example: A Biased Coin Biased coins are modeled using a Bernoulli
distribution, which describes probabilities for a binary variable.

\hypertarget{binomial-tests}{%
\section{Binomial Tests}\label{binomial-tests}}

A binomial test compares the number of successes observed in a given
number of trials with a hypothesized probability of success. The tes has
the null hypothesis that the real probability of success is equal to
some value denoted p.~and thr alternative hypothesis that is not equal
to p.~The test can also be performed with a one-sided alternative
hypothesis that the real probability of success is either greater than p
or that it is less than p.

A one-tailed test with a significance level of 0.05 will be used. You
roll the die 300 times and throw a total of 60 sixes. We cannot reject
the null hypothesis that the probability of rolling a six is 1/6. This
means that there is no evidence to prove that the die is not fair.

\begin{itemize}
\tightlist
\item
  Syntax: binom.test(nsuccesses, ntrials, p)
\end{itemize}

\begin{Shaded}
\begin{Highlighting}[]
\KeywordTok{binom.test}\NormalTok{(}\DecValTok{60}\NormalTok{, }\DecValTok{300}\NormalTok{, }\DecValTok{1}\OperatorTok{/}\DecValTok{6}\NormalTok{, }\DataTypeTok{alternative =} \StringTok{"greater"}\NormalTok{)}
\end{Highlighting}
\end{Shaded}

\begin{verbatim}
## 
##  Exact binomial test
## 
## data:  60 and 300
## number of successes = 60, number of trials = 300, p-value = 0.07299
## alternative hypothesis: true probability of success is greater than 0.1666667
## 95 percent confidence interval:
##  0.1626847 1.0000000
## sample estimates:
## probability of success 
##                    0.2
\end{verbatim}

\hypertarget{chi-square-tests}{%
\section{Chi Square Tests}\label{chi-square-tests}}

Chi Square tests are used to determine if two categorical varibales have
a significant correlation between them. The two variables are selected
from the same population. (male/female, red/green, yes/no). The
chi-square statistic is commonly used for testing relationships between
categorical variables.

\(\chi^2=\Sigma \frac {(obersverd - expected)^2}{expected}\)

\begin{itemize}
\item
  Syntax: chisp.test(data)
\item
  importing data, have a look at the table
\end{itemize}

\begin{Shaded}
\begin{Highlighting}[]
\NormalTok{data_frame<-}\StringTok{ }\KeywordTok{read.csv}\NormalTok{(}\StringTok{"https://goo.gl/j6lRXD"}\NormalTok{)}
\KeywordTok{table}\NormalTok{(data_frame}\OperatorTok{$}\NormalTok{treatment, data_frame}\OperatorTok{$}\NormalTok{improvement)}
\end{Highlighting}
\end{Shaded}

\begin{verbatim}
##              
##               improved not-improved
##   not-treated       26           29
##   treated           35           15
\end{verbatim}

\begin{itemize}
\tightlist
\item
  Example A: Here, we get a chi-squared value of 5.5569. Since we get a
  p-value less than the significance level of 0.05, we reject the null
  hypothesis and conclude that the two variables are in fact dependent.
\end{itemize}

\begin{Shaded}
\begin{Highlighting}[]
\KeywordTok{chisq.test}\NormalTok{(data_frame}\OperatorTok{$}\NormalTok{treatment, data_frame}\OperatorTok{$}\NormalTok{improvement, }\DataTypeTok{correct=}\OtherTok{FALSE}\NormalTok{)}
\end{Highlighting}
\end{Shaded}

\begin{verbatim}
## 
##  Pearson's Chi-squared test
## 
## data:  data_frame$treatment and data_frame$improvement
## X-squared = 5.5569, df = 1, p-value = 0.01841
\end{verbatim}

\begin{itemize}
\tightlist
\item
  Example B: Here, we get a high chi-squared value and a p-value of less
  than the 0.05 significance level. Therefore, we can reject the null
  hypothesis and conclude that carb and cyl have a significant
  relationship.
\end{itemize}

\begin{Shaded}
\begin{Highlighting}[]
\KeywordTok{data}\NormalTok{(}\StringTok{"mtcars"}\NormalTok{)}
\KeywordTok{table}\NormalTok{(mtcars}\OperatorTok{$}\NormalTok{carb, mtcars}\OperatorTok{$}\NormalTok{cyl)}
\end{Highlighting}
\end{Shaded}

\begin{verbatim}
##    
##     4 6 8
##   1 5 2 0
##   2 6 0 4
##   3 0 0 3
##   4 0 4 6
##   6 0 1 0
##   8 0 0 1
\end{verbatim}

\begin{Shaded}
\begin{Highlighting}[]
\KeywordTok{chisq.test}\NormalTok{(mtcars}\OperatorTok{$}\NormalTok{carb, mtcars}\OperatorTok{$}\NormalTok{cyl)}
\end{Highlighting}
\end{Shaded}

\begin{verbatim}
## Warning in chisq.test(mtcars$carb, mtcars$cyl): Chi-squared approximation may be
## incorrect
\end{verbatim}

\begin{verbatim}
## 
##  Pearson's Chi-squared test
## 
## data:  mtcars$carb and mtcars$cyl
## X-squared = 24.389, df = 10, p-value = 0.006632
\end{verbatim}

\hypertarget{confidence-intervals-and-p-values}{%
\section{Confidence Intervals and
P-values}\label{confidence-intervals-and-p-values}}

\emph{P-values quantify the probability of data being as or more extreme
than the data in hand, were the null hypothesis true.\\
}Confidence Intervals are more informative than p-values. We can reject
a null hypothesis if it lies outside of the confidence interval.

\hypertarget{gaussiannormal-distributions}{%
\section{Gaussian/Normal
Distributions}\label{gaussiannormal-distributions}}

Gaussian distribution (``the normal distribution'') is ubiquitous in
statistics. This distribution is based on the mean \(\mu\) and standard
deviation \(\sigma\) of the data.

\(x \sim Normal(\mu, \sigma^2)\)

\begin{itemize}
\tightlist
\item
  The Central Limit Theorem states that the population of all possible
  samples of size n from a populations= with mean \(\mu\) and variance
  (\(\sigma\) squared) approaches a normal distribution when n (sample
  size) approaches infinity. The central limit theorem makes the normal
  distribution extremely relevant.
\end{itemize}

The parameters of a statistical model that has been fit to a large
dataset will have lingering uncertainty, but this uncertainty will very
often be approximately normally distributed. This is why statisticians
so often assume that experimental measurements follow normal
distributions.

\(\theta\) = model parameter inferred from data
\(\theta \sim Normal(\mu,σ^2)\)

The goal of statistical inference is to determine the mean and standard
devistion of this distribution \(\mu\): best estimate of \(\theta\),
denoted \(\theta\) \(\sigma\): lingering uncertainty in \(\theta\):
affects confidence interval

\begin{itemize}
\tightlist
\item
  Example: Here, this normal/gaussian distribution results shows that
  the percentage of students scoring an 84\% or higher in the college
  entrance exam is 21.5\%
\end{itemize}

\begin{Shaded}
\begin{Highlighting}[]
\KeywordTok{pnorm}\NormalTok{(}\DecValTok{84}\NormalTok{, }\DataTypeTok{mean =} \DecValTok{72}\NormalTok{, }\DataTypeTok{sd =} \FloatTok{15.2}\NormalTok{, }\DataTypeTok{lower.tail =} \OtherTok{FALSE}\NormalTok{)}
\end{Highlighting}
\end{Shaded}

\begin{verbatim}
## [1] 0.2149176
\end{verbatim}

\hypertarget{standard-error-of-the-mean-sem}{%
\section{Standard Error of the Mean
(SEM)}\label{standard-error-of-the-mean-sem}}

The standard error of the mean (SEM) is a statistical term that measures
the accuracy with which a sample distribution represents a population by
using the standard deviation. In statistics, a smaple mean deviates from
the actual mean of a population - this deviation is the standard error
of the mean (SEM)

\(SEM = \sqrt{q(1-q)/N}\)

\begin{Shaded}
\begin{Highlighting}[]
\NormalTok{Input =(}\StringTok{"}
\StringTok{Stream                     Fish}
\StringTok{ Mill_Creek_1                76}
\StringTok{ Mill_Creek_2               102}
\StringTok{ North_Branch_Rock_Creek_1   12}
\StringTok{ North_Branch_Rock_Creek_2   39}
\StringTok{ Rock_Creek_1                55}
\StringTok{ Rock_Creek_2                93}
\StringTok{ Rock_Creek_3                98}
\StringTok{ Rock_Creek_4                53}
\StringTok{ Turkey_Branch              102}
\StringTok{"}\NormalTok{)}

\NormalTok{Data =}\StringTok{ }\KeywordTok{read.table}\NormalTok{(}\KeywordTok{textConnection}\NormalTok{(Input),}\DataTypeTok{header=}\OtherTok{TRUE}\NormalTok{)}
\end{Highlighting}
\end{Shaded}

\begin{itemize}
\tightlist
\item
  calculate standard error manually
\end{itemize}

\begin{Shaded}
\begin{Highlighting}[]
\KeywordTok{sd}\NormalTok{(Data}\OperatorTok{$}\NormalTok{Fish, }\DataTypeTok{na.rm=}\OtherTok{TRUE}\NormalTok{) }\OperatorTok{/}
\StringTok{  }\KeywordTok{sqrt}\NormalTok{(}\KeywordTok{length}\NormalTok{(Data}\OperatorTok{$}\NormalTok{Fish[}\OperatorTok{!}\KeywordTok{is.na}\NormalTok{(Data}\OperatorTok{$}\NormalTok{Fish)]))}
\end{Highlighting}
\end{Shaded}

\begin{verbatim}
## [1] 10.69527
\end{verbatim}

*install psych package in console and use describe function from package
for standard error. This function also works on while dataframes.

\begin{Shaded}
\begin{Highlighting}[]
\KeywordTok{describe}\NormalTok{(Data}\OperatorTok{$}\NormalTok{Fish, }
         \DataTypeTok{type=}\DecValTok{2}\NormalTok{)}
\end{Highlighting}
\end{Shaded}

\begin{verbatim}
##    vars n mean    sd median trimmed  mad min max range  skew kurtosis   se
## X1    1 9   70 32.09     76      70 34.1  12 102    90 -0.65    -0.69 10.7
\end{verbatim}

\hypertarget{z-tests}{%
\section{Z-Tests}\label{z-tests}}

The z-test function is based on the standard normal distribution and
creates confidence intervals and tests hypotheses for both one and two
sample problems.

\(z = \frac{x-\mu}{\sigma}\)

\emph{mu = a single number representing the value of the mean or
difference in means specified by the null hypothesis }sigma.x = a single
number representing the population standard deviation for x *sigma.y = a
single number representing the population standard deviation for y

*Syntax: z.test(x, y=NULL, alternative = ``two-sided'', mu = \(\theta\),
sigma.x=NULL, sigma.y=NULL, conf.level=0.95)

An R function called z.test() would be great for doing the kind of
testing in which you use z-scores in the hypothesis test. One problem:
that function does not exist in base R. Although you can find one in
other packages, it's easy enough to create one and learn a bit about R
programming in the process. The function would work like this:

Example: \emph{ID.data \textless{}- c(100,101,104, 109, 125, 116, 105,
108, 110) }z.test(IQ.data, 100, 15) \emph{z=1.733 Results: }one-tailed
probability = 0.042 *two-tailed probability = 0.084

Begin by creating the function name and its arguments:
\emph{z.test=function(x, \(\mu\), popvar)\{ }The first argument is the
vector of data, second is the population mean, third is the population
variance. Left curly bracket signifies that the remainder of the code is
what happens inside the function. Next, create a vector that will hold
the one-tailed probability of the z-score you will calculate:
\emph{one.tail.p \textless{}- NULL Then you calculate the z-score and
round it to three decimal places }z.score\textless{}-
round((mean(x)-mu)/popvar/sqrt(length(x))), 3)

\hypertarget{t-tests}{%
\section{T-Tests}\label{t-tests}}

A t-test is a statistical test which is used to compare the mean of two
groups of samples. It is therefore to evaluate whether the means of the
two sets of data are statistically significantly different from each
other.

The t.test() function produces a variety of t-tests. Unlike most
statistical packages, the default assume unequal variance and applies
the Welsh degrees of freedom modification.

A one-sample t-test is used to compare the mean of a population with a
theortetical value. An unpaired two sample t-test is used to compare the
mean of two independent samples. A paired t-test is used to compare the
means between two related groups of samples. The formula of a t-test
depends on the mean and standard deviation of the data being compared.

*One-Sample t-test: based on the theoretical mean(\mu), the set of
values with size n, the mean m, and the standard deviation \(\sigma\).
The degrees of freedom is found with (df=n-1).

\(t = \frac{m-\mu}{\frac{\sigma}{\sqrt{n}}}\)

*Independent two-sample t-test Let A and B represent the two groups to
compare. Let mA and mB represent the means of groups A and B,
respectively. Let nA and nB represent the sizes of group A and B
respectively.

\(t = \frac{mA - mB}{\sqrt{\frac{S^2}{nA} + \frac{S^2}{nB}}}\)

S\^{}2 is an estimator of the common variance of the two samples.

*Welch's T-Tests (or unequal variances t-test), is a two-sample location
test which is used to test the hypothesis that the two populations have
unequal means.

\begin{itemize}
\tightlist
\item
  independent two-sample t-test, where y is a numeric and x is a nibary
  factor
\end{itemize}

\begin{Shaded}
\begin{Highlighting}[]
\NormalTok{x=}\KeywordTok{rnorm}\NormalTok{(}\DecValTok{10}\NormalTok{)}
\NormalTok{y=}\KeywordTok{rnorm}\NormalTok{(}\DecValTok{10}\NormalTok{)}
\KeywordTok{t.test}\NormalTok{(y,x)  }\CommentTok{#Welch's t-test}
\end{Highlighting}
\end{Shaded}

\begin{verbatim}
## 
##  Welch Two Sample t-test
## 
## data:  y and x
## t = 0.055054, df = 12.403, p-value = 0.957
## alternative hypothesis: true difference in means is not equal to 0
## 95 percent confidence interval:
##  -0.9643282  1.0145094
## sample estimates:
##    mean of x    mean of y 
##  0.018022788 -0.007067778
\end{verbatim}

\begin{itemize}
\tightlist
\item
  one sample t-test, with a null hypothesis (Ho) mu=3
\end{itemize}

\begin{Shaded}
\begin{Highlighting}[]
\KeywordTok{t.test}\NormalTok{(y,}\DataTypeTok{mu=}\DecValTok{3}\NormalTok{) }
\end{Highlighting}
\end{Shaded}

\begin{verbatim}
## 
##  One Sample t-test
## 
## data:  y
## t = -7.1566, df = 9, p-value = 5.328e-05
## alternative hypothesis: true mean is not equal to 3
## 95 percent confidence interval:
##  -0.9245609  0.9606065
## sample estimates:
##  mean of x 
## 0.01802279
\end{verbatim}

*Mann-Whitney-Wilcoxon Test Two data samples are independent if they
come from distinct populations and the samples do not affect each other.
Using the Mann-Whitney-Wilcoxon Test, we can decide whether the
population distribution are identical without assuming them to follow
the normal distribution.

Example: At the 0.05 significance level, we conclude that the gas
mileage data of manual and automatic transmissions in mtcar are
nonidentical populations.

\begin{Shaded}
\begin{Highlighting}[]
\NormalTok{mtcars}\OperatorTok{$}\NormalTok{mpg}
\end{Highlighting}
\end{Shaded}

\begin{verbatim}
##  [1] 21.0 21.0 22.8 21.4 18.7 18.1 14.3 24.4 22.8 19.2 17.8 16.4 17.3 15.2 10.4
## [16] 10.4 14.7 32.4 30.4 33.9 21.5 15.5 15.2 13.3 19.2 27.3 26.0 30.4 15.8 19.7
## [31] 15.0 21.4
\end{verbatim}

\begin{Shaded}
\begin{Highlighting}[]
\KeywordTok{wilcox.test}\NormalTok{(mpg }\OperatorTok{~}\StringTok{ }\NormalTok{am, }\DataTypeTok{data=}\NormalTok{mtcars)}
\end{Highlighting}
\end{Shaded}

\begin{verbatim}
## Warning in wilcox.test.default(x = c(21.4, 18.7, 18.1, 14.3, 24.4, 22.8, :
## cannot compute exact p-value with ties
\end{verbatim}

\begin{verbatim}
## 
##  Wilcoxon rank sum test with continuity correction
## 
## data:  mpg by am
## W = 42, p-value = 0.001871
## alternative hypothesis: true location shift is not equal to 0
\end{verbatim}

\hypertarget{qq-plots}{%
\section{QQ Plots}\label{qq-plots}}

QQ Plots are used to visually test whether data follows an expected
distribution. They are used to verify that data used in a t-test is
actually normally distributed. QQ plots are not useful on small
datasets.

\begin{itemize}
\tightlist
\item
  in console: install.packages(``car'')
\end{itemize}

\begin{Shaded}
\begin{Highlighting}[]
\NormalTok{my_data <-}\StringTok{ }\NormalTok{ToothGrowth}

\KeywordTok{qqnorm}\NormalTok{(my_data}\OperatorTok{$}\NormalTok{len, }\DataTypeTok{pch =} \DecValTok{1}\NormalTok{, }\DataTypeTok{frame =} \OtherTok{FALSE}\NormalTok{)}
\KeywordTok{qqline}\NormalTok{(my_data}\OperatorTok{$}\NormalTok{len, }\DataTypeTok{col =} \StringTok{"steelblue"}\NormalTok{, }\DataTypeTok{lwd=}\DecValTok{2}\NormalTok{)}
\end{Highlighting}
\end{Shaded}

\includegraphics{PracticalStatisticsR_files/figure-latex/unnamed-chunk-31-1.pdf}

\begin{Shaded}
\begin{Highlighting}[]
\KeywordTok{qqPlot}\NormalTok{(my_data}\OperatorTok{$}\NormalTok{len)}
\end{Highlighting}
\end{Shaded}

\includegraphics{PracticalStatisticsR_files/figure-latex/unnamed-chunk-31-2.pdf}

\begin{verbatim}
## [1] 23  1
\end{verbatim}

\hypertarget{the-correlation-coefficient-r}{%
\section{The Correlation Coefficient
(r)}\label{the-correlation-coefficient-r}}

The main result of a correlation is called the correlation coefficient
(``r''). It ranges from -1 to 1. The closer r is to -1 or 1, the more
closely the two variables are related. If r is close to 0, it means
there is no relationship between the variables. \emph{When r=0, the two
variables are independent. }When r=+1/-1, the two variables share a
deterministic linear relationship.

\hypertarget{the-coefficient-of-determination-r-squared}{%
\section{The Coefficient of Determination (r
squared)}\label{the-coefficient-of-determination-r-squared}}

R-squared is always between 0 and 1. It is commonly interpreted as the
fraction of variance in y explained by x (or the other way around). It
is a goodness-of-fit measure for linear regression models. This
statistics indicates a percaentage of the variance in the dependent
variable that the independent variables explain collectively. T-squared
measures the strength of the relationship between your model and the
dependent variable on a convenient 0-100\% scale.

After fitting a linear regression modelto the data (by using a
least-squares regression line), we need to determine how well the model
fits the data.

Assessing Goodness-of-Fit in a Regression Model *Linear Regression
identifies the equation that produces the smallest difference between
all of the observes values and their fitted values. Linear regression
finds thr smallest sum of squared residuals that is possible for the
dataset (least-squares regression line).

R-Squared is the perectnage of the dependent variable variation that a
linear model explains.

\(R^2 = \frac{variance explained by model}{total variance}\)

\begin{itemize}
\item
  0\% or and r-squared close to 0.000 represents a model that does not
  explain any of the variation in the resprinse (dependent) variable
  around its mean. The mean of the dependent variable predicts the
  dependent variable as well as the regression model.
\item
  100\% or an r-squared of 1.00 represents a model that explains all of
  the variation in the response variable around its mean.
\item
  Example: The coefficient of determination (r squared) of the simple
  linear regression model for the data set below is 0.81146. This means
  with an r-squared of 0.81146, then approximately 81.15\% of the
  observed variation can be explained by the model's inputs.
\end{itemize}

\begin{Shaded}
\begin{Highlighting}[]
\CommentTok{#We apply the lm function to a formula that desribes the variable eruptions by the variable waiting, and save the linear regression model in a new variable eruption.lm.  }
\NormalTok{eruption.lm =}\StringTok{ }\KeywordTok{lm}\NormalTok{(eruptions }\OperatorTok{~}\StringTok{ }\NormalTok{waiting, }\DataTypeTok{data=}\NormalTok{faithful)}
\KeywordTok{summary}\NormalTok{(eruption.lm)}\OperatorTok{$}\NormalTok{r.squared }
\end{Highlighting}
\end{Shaded}

\begin{verbatim}
## [1] 0.8114608
\end{verbatim}

\begin{Shaded}
\begin{Highlighting}[]
\KeywordTok{summary}\NormalTok{(eruption.lm)}
\end{Highlighting}
\end{Shaded}

\begin{verbatim}
## 
## Call:
## lm(formula = eruptions ~ waiting, data = faithful)
## 
## Residuals:
##      Min       1Q   Median       3Q      Max 
## -1.29917 -0.37689  0.03508  0.34909  1.19329 
## 
## Coefficients:
##              Estimate Std. Error t value Pr(>|t|)    
## (Intercept) -1.874016   0.160143  -11.70   <2e-16 ***
## waiting      0.075628   0.002219   34.09   <2e-16 ***
## ---
## Signif. codes:  0 '***' 0.001 '**' 0.01 '*' 0.05 '.' 0.1 ' ' 1
## 
## Residual standard error: 0.4965 on 270 degrees of freedom
## Multiple R-squared:  0.8115, Adjusted R-squared:  0.8108 
## F-statistic:  1162 on 1 and 270 DF,  p-value: < 2.2e-16
\end{verbatim}

\hypertarget{power-analysis}{%
\section{Power Analysis}\label{power-analysis}}

The power of a statistical test is the probability that the test will
reject a false null hypothesis. Power analysis allows us to determine
the sample size required to detect an effect of a given size with a
given degree of confidence. It also allows us to determine the
probability of detecting an effect of a given size with a given level of
confidence, under sample size constraints.

*Example: The function tells us we should flip the coin 22.55126 times,
which would round up to 23. Always round sample size estimates up in
power analyses. If we are correct that our coin lands heads 75\% of the
time, we need to flip at least 23 times to have an 80\% chance of
correctly rejecting the null hypothesis at the 0.05 significance level.

\begin{itemize}
\tightlist
\item
  Syntax:
\end{itemize}

\begin{Shaded}
\begin{Highlighting}[]
\CommentTok{#pwr.2p.test -- two proportions (equal n)}
\CommentTok{#pwr.2p2n.test -- two proportions(unequal n)}
\CommentTok{#pwr.anova.test -- balanced one-way ANOVA}
\CommentTok{#pwr.chisq.test  -- chi-square test}
\CommentTok{#pwr.f2.test -- general linear model}
\CommentTok{#pwr.p.test -- proportion (one sample)}
\CommentTok{#pwr.r.test -- correlation}
\CommentTok{#pwr.t.test -- t-tests (one sample, 2 sample, paired)}
\CommentTok{#pwr.t2n.test   -- t-test (two samples with unequal n)}
\end{Highlighting}
\end{Shaded}

\begin{Shaded}
\begin{Highlighting}[]
\KeywordTok{pwr.p.test}\NormalTok{(}\DataTypeTok{h=}\KeywordTok{ES.h}\NormalTok{(}\DataTypeTok{p1=}\FloatTok{0.75}\NormalTok{, }\DataTypeTok{p2=}\FloatTok{0.50}\NormalTok{), }
           \DataTypeTok{sig.level =} \FloatTok{0.05}\NormalTok{,}
           \DataTypeTok{power =} \FloatTok{0.80}\NormalTok{, }
           \DataTypeTok{alternative =} \StringTok{"greater"}\NormalTok{)}
\end{Highlighting}
\end{Shaded}

\begin{verbatim}
## 
##      proportion power calculation for binomial distribution (arcsine transformation) 
## 
##               h = 0.5235988
##               n = 22.55126
##       sig.level = 0.05
##           power = 0.8
##     alternative = greater
\end{verbatim}

\begin{Shaded}
\begin{Highlighting}[]
\NormalTok{p.out <-}\StringTok{ }\KeywordTok{pwr.p.test}\NormalTok{(}\DataTypeTok{h =} \KeywordTok{ES.h}\NormalTok{(}\DataTypeTok{p1 =} \FloatTok{0.75}\NormalTok{, }\DataTypeTok{p2 =} \FloatTok{0.50}\NormalTok{),}
                    \DataTypeTok{sig.level =} \FloatTok{0.05}\NormalTok{,}
                    \DataTypeTok{power =} \FloatTok{0.80}\NormalTok{,}
                    \DataTypeTok{alternative =} \StringTok{"greater"}\NormalTok{)}
\KeywordTok{plot}\NormalTok{(p.out)}
\end{Highlighting}
\end{Shaded}

\includegraphics{PracticalStatisticsR_files/figure-latex/unnamed-chunk-34-1.pdf}
Example: What is the power of our test if we flip the coin 40 times and
lower our Type I error tolerance to 0.01? Notice we leave out the power
argument, add n = 40, and change sig.level = 0.01:

The power of our test is now about 84\%. If we wish to assume a
``two-sided'' alternative, we can simply leave it out of the function.
Notice how our power estimate drops below 80\% when we do this.

\begin{Shaded}
\begin{Highlighting}[]
\KeywordTok{pwr.p.test}\NormalTok{(}\DataTypeTok{h =} \KeywordTok{ES.h}\NormalTok{(}\DataTypeTok{p1 =} \FloatTok{0.75}\NormalTok{, }\DataTypeTok{p2 =} \FloatTok{0.50}\NormalTok{),}
           \DataTypeTok{sig.level =} \FloatTok{0.01}\NormalTok{,}
           \DataTypeTok{n =} \DecValTok{40}\NormalTok{,}
           \DataTypeTok{alternative =} \StringTok{"greater"}\NormalTok{)}
\end{Highlighting}
\end{Shaded}

\begin{verbatim}
## 
##      proportion power calculation for binomial distribution (arcsine transformation) 
## 
##               h = 0.5235988
##               n = 40
##       sig.level = 0.01
##           power = 0.8377325
##     alternative = greater
\end{verbatim}

\begin{Shaded}
\begin{Highlighting}[]
\KeywordTok{pwr.p.test}\NormalTok{(}\DataTypeTok{h =} \KeywordTok{ES.h}\NormalTok{(}\DataTypeTok{p1 =} \FloatTok{0.75}\NormalTok{, }\DataTypeTok{p2 =} \FloatTok{0.50}\NormalTok{),}
           \DataTypeTok{sig.level =} \FloatTok{0.01}\NormalTok{,}
           \DataTypeTok{n =} \DecValTok{40}\NormalTok{)}
\end{Highlighting}
\end{Shaded}

\begin{verbatim}
## 
##      proportion power calculation for binomial distribution (arcsine transformation) 
## 
##               h = 0.5235988
##               n = 40
##       sig.level = 0.01
##           power = 0.7690434
##     alternative = two.sided
\end{verbatim}

\hypertarget{anova-tests-analysis-of-variance}{%
\section{ANOVA tests (Analysis of
Variance)}\label{anova-tests-analysis-of-variance}}

ANOVA is a statistical test for estimating how a quantitative dependent
variable changes according to the levels of one or more categorical
independent variables. ANOVA tests whether there is a difference in
means of the groups at each level of the independent variable.

Example: You can check the levels of poison with the following. You
should see three character values because you convert them in factor
with the mutate verb.

\begin{Shaded}
\begin{Highlighting}[]
\NormalTok{PATH<-}\StringTok{ "https://raw.githubusercontent.com/guru99-edu/R-Programming/master/poisons.csv"}
\NormalTok{df <-}\StringTok{ }\KeywordTok{read.csv}\NormalTok{(PATH) }\OperatorTok
\KeywordTok{select}\NormalTok{(}\OperatorTok{-}\NormalTok{X) }\OperatorTok\StringTok{ }
\KeywordTok{mutate}\NormalTok{(}\DataTypeTok{poison =} \KeywordTok{factor}\NormalTok{(poison, }\DataTypeTok{ordered =} \OtherTok{TRUE}\NormalTok{))}
\KeywordTok{glimpse}\NormalTok{(df)}
\end{Highlighting}
\end{Shaded}

\begin{verbatim}
## Rows: 48
## Columns: 3
## $ time   <dbl> 0.31, 0.45, 0.46, 0.43, 0.36, 0.29, 0.40, 0.23, 0.22, 0.21, ...
## $ poison <ord> 1, 1, 1, 1, 2, 2, 2, 2, 3, 3, 3, 3, 1, 1, 1, 1, 2, 2, 2, 2, ...
## $ treat  <chr> "A", "A", "A", "A", "A", "A", "A", "A", "A", "A", "A", "A", ...
\end{verbatim}

\begin{Shaded}
\begin{Highlighting}[]
\KeywordTok{levels}\NormalTok{(df}\OperatorTok{$}\NormalTok{poison)}
\end{Highlighting}
\end{Shaded}

\begin{verbatim}
## [1] "1" "2" "3"
\end{verbatim}

\begin{itemize}
\tightlist
\item
  Compute the mean and the standard deviation
\end{itemize}

\begin{Shaded}
\begin{Highlighting}[]
\NormalTok{df }\OperatorTok
\StringTok{  }\KeywordTok{group_by}\NormalTok{(poison)}\OperatorTok
\StringTok{  }\KeywordTok{summarise}\NormalTok{(}
    \DataTypeTok{count_poison =} \KeywordTok{n}\NormalTok{(), }
    \DataTypeTok{mean_time =} \KeywordTok{mean}\NormalTok{(time, }\DataTypeTok{na.rm =} \OtherTok{TRUE}\NormalTok{),}
    \DataTypeTok{sd_time =} \KeywordTok{sd}\NormalTok{(time, }\DataTypeTok{na.rm=}\OtherTok{TRUE}\NormalTok{)}
\NormalTok{  )}
\end{Highlighting}
\end{Shaded}

\begin{verbatim}
## # A tibble: 3 x 4
##   poison count_poison mean_time sd_time
##   <ord>         <int>     <dbl>   <dbl>
## 1 1                16     0.618  0.209 
## 2 2                16     0.544  0.289 
## 3 3                16     0.276  0.0623
\end{verbatim}

\begin{itemize}
\tightlist
\item
  Graphically check if there is a difference between the distribution.
\end{itemize}

\begin{Shaded}
\begin{Highlighting}[]
\KeywordTok{ggplot}\NormalTok{(df, }\KeywordTok{aes}\NormalTok{(}\DataTypeTok{x =}\NormalTok{ poison, }\DataTypeTok{y =}\NormalTok{ time, }\DataTypeTok{fill =}\NormalTok{ poison)) }\OperatorTok{+}
\StringTok{    }\KeywordTok{geom_boxplot}\NormalTok{() }\OperatorTok{+}
\StringTok{    }\KeywordTok{geom_jitter}\NormalTok{(}\DataTypeTok{shape =} \DecValTok{15}\NormalTok{,}
        \DataTypeTok{color =} \StringTok{"steelblue"}\NormalTok{,}
        \DataTypeTok{position =} \KeywordTok{position_jitter}\NormalTok{(}\FloatTok{0.21}\NormalTok{)) }\OperatorTok{+}
\StringTok{    }\KeywordTok{theme_classic}\NormalTok{()}
\end{Highlighting}
\end{Shaded}

\includegraphics{PracticalStatisticsR_files/figure-latex/unnamed-chunk-38-1.pdf}
* Run the one-way ANOVA test with the command aov. * Syntax:
aov(formula, data)

\begin{Shaded}
\begin{Highlighting}[]
\NormalTok{anova_one_way <-}\StringTok{ }\KeywordTok{aov}\NormalTok{(time}\OperatorTok{~}\NormalTok{poison, }\DataTypeTok{data =}\NormalTok{ df)}
\KeywordTok{summary}\NormalTok{(anova_one_way)}
\end{Highlighting}
\end{Shaded}

\begin{verbatim}
##             Df Sum Sq Mean Sq F value   Pr(>F)    
## poison       2  1.033  0.5165   11.79 7.66e-05 ***
## Residuals   45  1.972  0.0438                     
## ---
## Signif. codes:  0 '***' 0.001 '**' 0.01 '*' 0.05 '.' 0.1 ' ' 1
\end{verbatim}

\begin{itemize}
\tightlist
\item
  Tukeys Test:
\end{itemize}

\begin{Shaded}
\begin{Highlighting}[]
\KeywordTok{TukeyHSD}\NormalTok{(anova_one_way)}
\end{Highlighting}
\end{Shaded}

\begin{verbatim}
##   Tukey multiple comparisons of means
##     95% family-wise confidence level
## 
## Fit: aov(formula = time ~ poison, data = df)
## 
## $poison
##          diff        lwr         upr     p adj
## 2-1 -0.073125 -0.2525046  0.10625464 0.5881654
## 3-1 -0.341250 -0.5206296 -0.16187036 0.0000971
## 3-2 -0.268125 -0.4475046 -0.08874536 0.0020924
\end{verbatim}

*ANOVA two-way test

\begin{Shaded}
\begin{Highlighting}[]
\NormalTok{anova_two_way <-}\StringTok{ }\KeywordTok{aov}\NormalTok{(time}\OperatorTok{~}\NormalTok{poison }\OperatorTok{+}\StringTok{ }\NormalTok{treat, }\DataTypeTok{data =}\NormalTok{ df)}
\KeywordTok{summary}\NormalTok{(anova_two_way)}
\end{Highlighting}
\end{Shaded}

\begin{verbatim}
##             Df Sum Sq Mean Sq F value  Pr(>F)    
## poison       2 1.0330  0.5165   20.64 5.7e-07 ***
## treat        3 0.9212  0.3071   12.27 6.7e-06 ***
## Residuals   42 1.0509  0.0250                    
## ---
## Signif. codes:  0 '***' 0.001 '**' 0.01 '*' 0.05 '.' 0.1 ' ' 1
\end{verbatim}

\hypertarget{nonlinear-regression}{%
\section{Nonlinear Regression}\label{nonlinear-regression}}

\begin{itemize}
\tightlist
\item
  Load the data
\end{itemize}

\begin{Shaded}
\begin{Highlighting}[]
\KeywordTok{theme_set}\NormalTok{(}\KeywordTok{theme_classic}\NormalTok{())}
\KeywordTok{data}\NormalTok{(}\StringTok{"Boston"}\NormalTok{, }\DataTypeTok{package =} \StringTok{"MASS"}\NormalTok{)}
\end{Highlighting}
\end{Shaded}

\begin{itemize}
\tightlist
\item
  Split the data into training and test set
\end{itemize}

\begin{Shaded}
\begin{Highlighting}[]
\KeywordTok{set.seed}\NormalTok{(}\DecValTok{123}\NormalTok{)}
\NormalTok{training.samples <-}\StringTok{ }\NormalTok{Boston}\OperatorTok{$}\NormalTok{medv }\OperatorTok
\StringTok{  }\KeywordTok{createDataPartition}\NormalTok{(}\DataTypeTok{p =} \FloatTok{0.8}\NormalTok{, }\DataTypeTok{list =} \OtherTok{FALSE}\NormalTok{)}
\NormalTok{train.data  <-}\StringTok{ }\NormalTok{Boston[training.samples, ]}
\NormalTok{test.data <-}\StringTok{ }\NormalTok{Boston[}\OperatorTok{-}\NormalTok{training.samples, ]}
\end{Highlighting}
\end{Shaded}

\begin{itemize}
\tightlist
\item
  First, visualize the scatter plot of the medv vs lstat variables:
\end{itemize}

\begin{Shaded}
\begin{Highlighting}[]
\KeywordTok{ggplot}\NormalTok{(train.data, }\KeywordTok{aes}\NormalTok{(lstat, medv) ) }\OperatorTok{+}
\StringTok{  }\KeywordTok{geom_point}\NormalTok{() }\OperatorTok{+}
\StringTok{  }\KeywordTok{stat_smooth}\NormalTok{()}
\end{Highlighting}
\end{Shaded}

\includegraphics{PracticalStatisticsR_files/figure-latex/unnamed-chunk-44-1.pdf}

The standard linear regression model equation can be writted as:

\(medv = b0+b1*lstat\)

Compute the linear regression model: * Build the model

\begin{Shaded}
\begin{Highlighting}[]
\NormalTok{model <-}\StringTok{ }\KeywordTok{lm}\NormalTok{(medv }\OperatorTok{~}\StringTok{ }\NormalTok{lstat, }\DataTypeTok{data =}\NormalTok{ train.data)}
\end{Highlighting}
\end{Shaded}

\begin{itemize}
\tightlist
\item
  Make predictions
\end{itemize}

\begin{Shaded}
\begin{Highlighting}[]
\NormalTok{predictions <-}\StringTok{ }\NormalTok{model }\OperatorTok\StringTok{ }\KeywordTok{predict}\NormalTok{(test.data)}
\end{Highlighting}
\end{Shaded}

\begin{itemize}
\tightlist
\item
  Model performance
\end{itemize}

\begin{Shaded}
\begin{Highlighting}[]
\KeywordTok{data.frame}\NormalTok{(}
  \DataTypeTok{RMSE =} \KeywordTok{RMSE}\NormalTok{(predictions, test.data}\OperatorTok{$}\NormalTok{medv),}
  \DataTypeTok{R2 =} \KeywordTok{R2}\NormalTok{(predictions, test.data}\OperatorTok{$}\NormalTok{medv)}
\NormalTok{)}
\end{Highlighting}
\end{Shaded}

\begin{verbatim}
##       RMSE       R2
## 1 6.503817 0.513163
\end{verbatim}

\begin{Shaded}
\begin{Highlighting}[]
\KeywordTok{ggplot}\NormalTok{(train.data, }\KeywordTok{aes}\NormalTok{(lstat, medv) ) }\OperatorTok{+}
\StringTok{  }\KeywordTok{geom_point}\NormalTok{() }\OperatorTok{+}
\StringTok{  }\KeywordTok{stat_smooth}\NormalTok{(}\DataTypeTok{method =}\NormalTok{ lm, }\DataTypeTok{formula =}\NormalTok{ y }\OperatorTok{~}\StringTok{ }\NormalTok{x)}
\end{Highlighting}
\end{Shaded}

\includegraphics{PracticalStatisticsR_files/figure-latex/unnamed-chunk-47-1.pdf}

\hypertarget{polynomial-regression}{%
\section{Polynomial Regression}\label{polynomial-regression}}

Polynomial regression: adds polynomial or quadratic terms to the
regression equations

\begin{Shaded}
\begin{Highlighting}[]
\KeywordTok{lm}\NormalTok{(medv }\OperatorTok{~}\StringTok{ }\NormalTok{lstat }\OperatorTok{+}\StringTok{ }\KeywordTok{I}\NormalTok{(lstat}\OperatorTok{^}\DecValTok{2}\NormalTok{), }\DataTypeTok{data =}\NormalTok{ train.data)}
\end{Highlighting}
\end{Shaded}

\begin{verbatim}
## 
## Call:
## lm(formula = medv ~ lstat + I(lstat^2), data = train.data)
## 
## Coefficients:
## (Intercept)        lstat   I(lstat^2)  
##     42.5736      -2.2673       0.0412
\end{verbatim}

\begin{Shaded}
\begin{Highlighting}[]
\CommentTok{#or}
\KeywordTok{lm}\NormalTok{(medv }\OperatorTok{~}\StringTok{ }\KeywordTok{poly}\NormalTok{(lstat, }\DecValTok{2}\NormalTok{, }\DataTypeTok{raw =} \OtherTok{TRUE}\NormalTok{), }\DataTypeTok{data =}\NormalTok{ train.data)}
\end{Highlighting}
\end{Shaded}

\begin{verbatim}
## 
## Call:
## lm(formula = medv ~ poly(lstat, 2, raw = TRUE), data = train.data)
## 
## Coefficients:
##                 (Intercept)  poly(lstat, 2, raw = TRUE)1  
##                     42.5736                      -2.2673  
## poly(lstat, 2, raw = TRUE)2  
##                      0.0412
\end{verbatim}

\begin{Shaded}
\begin{Highlighting}[]
\KeywordTok{lm}\NormalTok{(medv }\OperatorTok{~}\StringTok{ }\KeywordTok{poly}\NormalTok{(lstat, }\DecValTok{6}\NormalTok{, }\DataTypeTok{raw =} \OtherTok{TRUE}\NormalTok{), }\DataTypeTok{data =}\NormalTok{ train.data) }\OperatorTok
\StringTok{  }\KeywordTok{summary}\NormalTok{()}
\end{Highlighting}
\end{Shaded}

\begin{verbatim}
## 
## Call:
## lm(formula = medv ~ poly(lstat, 6, raw = TRUE), data = train.data)
## 
## Residuals:
##      Min       1Q   Median       3Q      Max 
## -13.1962  -3.1527  -0.7655   2.0404  26.7661 
## 
## Coefficients:
##                               Estimate Std. Error t value Pr(>|t|)    
## (Intercept)                  7.788e+01  6.844e+00  11.379  < 2e-16 ***
## poly(lstat, 6, raw = TRUE)1 -1.767e+01  3.569e+00  -4.952 1.08e-06 ***
## poly(lstat, 6, raw = TRUE)2  2.417e+00  6.779e-01   3.566 0.000407 ***
## poly(lstat, 6, raw = TRUE)3 -1.761e-01  6.105e-02  -2.885 0.004121 ** 
## poly(lstat, 6, raw = TRUE)4  6.845e-03  2.799e-03   2.446 0.014883 *  
## poly(lstat, 6, raw = TRUE)5 -1.343e-04  6.290e-05  -2.136 0.033323 *  
## poly(lstat, 6, raw = TRUE)6  1.047e-06  5.481e-07   1.910 0.056910 .  
## ---
## Signif. codes:  0 '***' 0.001 '**' 0.01 '*' 0.05 '.' 0.1 ' ' 1
## 
## Residual standard error: 5.188 on 400 degrees of freedom
## Multiple R-squared:  0.6845, Adjusted R-squared:  0.6798 
## F-statistic: 144.6 on 6 and 400 DF,  p-value: < 2.2e-16
\end{verbatim}

\begin{itemize}
\tightlist
\item
  Build the model
\end{itemize}

\begin{Shaded}
\begin{Highlighting}[]
\NormalTok{model <-}\StringTok{ }\KeywordTok{lm}\NormalTok{(medv }\OperatorTok{~}\StringTok{ }\KeywordTok{poly}\NormalTok{(lstat, }\DecValTok{5}\NormalTok{, }\DataTypeTok{raw =} \OtherTok{TRUE}\NormalTok{), }\DataTypeTok{data =}\NormalTok{ train.data)}
\end{Highlighting}
\end{Shaded}

\begin{itemize}
\tightlist
\item
  Make predictions
\end{itemize}

\begin{Shaded}
\begin{Highlighting}[]
\NormalTok{predictions <-}\StringTok{ }\NormalTok{model }\OperatorTok\StringTok{ }\KeywordTok{predict}\NormalTok{(test.data)}
\end{Highlighting}
\end{Shaded}

\begin{itemize}
\tightlist
\item
  Model performance
\end{itemize}

\begin{Shaded}
\begin{Highlighting}[]
\KeywordTok{data.frame}\NormalTok{(}
  \DataTypeTok{RMSE =} \KeywordTok{RMSE}\NormalTok{(predictions, test.data}\OperatorTok{$}\NormalTok{medv),}
  \DataTypeTok{R2 =} \KeywordTok{R2}\NormalTok{(predictions, test.data}\OperatorTok{$}\NormalTok{medv)}
\NormalTok{)}
\end{Highlighting}
\end{Shaded}

\begin{verbatim}
##       RMSE        R2
## 1 5.270374 0.6829474
\end{verbatim}

\begin{Shaded}
\begin{Highlighting}[]
\KeywordTok{ggplot}\NormalTok{(train.data, }\KeywordTok{aes}\NormalTok{(lstat, medv) ) }\OperatorTok{+}
\StringTok{  }\KeywordTok{geom_point}\NormalTok{() }\OperatorTok{+}
\StringTok{  }\KeywordTok{stat_smooth}\NormalTok{(}\DataTypeTok{method =}\NormalTok{ lm, }\DataTypeTok{formula =}\NormalTok{ y }\OperatorTok{~}\StringTok{ }\KeywordTok{poly}\NormalTok{(x, }\DecValTok{5}\NormalTok{, }\DataTypeTok{raw =} \OtherTok{TRUE}\NormalTok{))}
\end{Highlighting}
\end{Shaded}

\includegraphics{PracticalStatisticsR_files/figure-latex/unnamed-chunk-51-1.pdf}

\hypertarget{log-transformation}{%
\subsection{Log Transformation}\label{log-transformation}}

Log transformation: when you have a non-linear relationship, you can try
a logarithmic transormation of the predictor variables:

\begin{itemize}
\tightlist
\item
  Build the model
\end{itemize}

\begin{Shaded}
\begin{Highlighting}[]
\NormalTok{model <-}\StringTok{ }\KeywordTok{lm}\NormalTok{(medv }\OperatorTok{~}\StringTok{ }\KeywordTok{log}\NormalTok{(lstat), }\DataTypeTok{data =}\NormalTok{ train.data)}
\end{Highlighting}
\end{Shaded}

\begin{itemize}
\tightlist
\item
  Make predictions
\end{itemize}

\begin{Shaded}
\begin{Highlighting}[]
\NormalTok{predictions <-}\StringTok{ }\NormalTok{model }\OperatorTok\StringTok{ }\KeywordTok{predict}\NormalTok{(test.data)}
\end{Highlighting}
\end{Shaded}

\begin{itemize}
\tightlist
\item
  Model performance
\end{itemize}

\begin{Shaded}
\begin{Highlighting}[]
\KeywordTok{data.frame}\NormalTok{(}
  \DataTypeTok{RMSE =} \KeywordTok{RMSE}\NormalTok{(predictions, test.data}\OperatorTok{$}\NormalTok{medv),}
  \DataTypeTok{R2 =} \KeywordTok{R2}\NormalTok{(predictions, test.data}\OperatorTok{$}\NormalTok{medv)}
\NormalTok{)}
\end{Highlighting}
\end{Shaded}

\begin{verbatim}
##       RMSE        R2
## 1 5.467124 0.6570091
\end{verbatim}

\begin{Shaded}
\begin{Highlighting}[]
\KeywordTok{ggplot}\NormalTok{(train.data, }\KeywordTok{aes}\NormalTok{(lstat, medv) ) }\OperatorTok{+}
\StringTok{  }\KeywordTok{geom_point}\NormalTok{() }\OperatorTok{+}
\StringTok{  }\KeywordTok{stat_smooth}\NormalTok{(}\DataTypeTok{method =}\NormalTok{ lm, }\DataTypeTok{formula =}\NormalTok{ y }\OperatorTok{~}\StringTok{ }\KeywordTok{log}\NormalTok{(x))}
\end{Highlighting}
\end{Shaded}

\includegraphics{PracticalStatisticsR_files/figure-latex/unnamed-chunk-54-1.pdf}

\end{document}
